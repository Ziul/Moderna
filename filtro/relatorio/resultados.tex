%Deve conter a tabela verdade obtida em laboratorio comentada
%(principalmente nos estados ambíguos)
%Deve apresentar também as dificuldades encontradas no experimento

%Apresentação dos resultados referentes à confecção do modelo em estudo e medidas realizadas, 
%\begin{SCfigure}[1][h] %%Cuidado..ela não se mantem no lugar
%  \centering
%  \includegraphics[width=4cm]{./fts/unb}
%  \caption{Legenda da figura \ref{caption1}. Coloca qualquer coisa.}
%  \label{caption1}
%\end{SCfigure}
%em forma de texto, tabelas e gráficos, como visto na figura \ref{caption} ou na figura \ref{caption1}. Também pode ser incluida nos anexos, como a figura \ref{unb}.


O grupo conseguiu com este experimento uma grande carga de conhecimento. Não em decorrer da conversão analógico-digital, mas por que fomos responsáveis por levantar a biblioteca do LCD que foi usado especificamente para este experimento. Como observado na função principal, o programa é bem simples, porém o peso da biblioteca do LCD é visível no arquivo final, visto que o executável grava 7378 bytes no MSP, o que é um tamanho bem elevado para apenas uma conversão analógica-digital. 

Outro ponto de grande aprendizado com este experimento foi a reação de filtros e como implementa-los em microcontroladores, visto que foi aplicado no experimento um filtro que tem um comportamento próximo à um filtro passa baixa.
