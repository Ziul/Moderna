\section{Discussão}\label{discurso}


O experimento de Young é um dos enigmas da ciência moderna. Famoso por ser uma das forma de demonstração da dualidade do fóton - como visto na sessão \ref{dualidade}. Foi possível observar através deste experimento que a reação que ocorre com as ondas ao se passar por frestas com valores próximos do seu comprimento de onda, também ocorrer com a luz. Em especial neste experimento ao laser. Esta interferência observada em um sinal de luz rigorosamente orientado como o laser nos remete ao fator de que, há no fóton uma onda que apresenta uma difração que se enquadra no caso na Difração de Fraunhofer. Características já fundamentadas e bem parametrizadas, observadas de demais fontes de ondas, como é em \cite{ondas}, onde há inclusive uma demonstração para obter a equação que caracteriza a intensidade de onda observada na superfície, descrita na equação \ref{sinal}.

%-------------------------------------------------------------------------------

\begin{equation}\label{sinal}
	I(\beta)=I_0 \frac{sin^2\beta}{\beta^2}
\end{equation}
Sendo $I_0$ o valor de intensidade máxima da luz observado no intervalo $\beta$, obtendo assim valores mínimos para $\beta=\pm n \pi$ e valores máximos relativos nas raízes de $\beta=tg(\beta)$.

%-------------------------------------------------------------------------------

\section{Conclusão}\label{conclusion}

A explicação de por que este efeito ocorrem  em fótons ainda é um mistério para a ciência, sendo a dualidade uma das alternativas encontradas para se explicar o efeito observado, porém é um resultado empírico oriundo de experimentos onde apenas os efeitos são observados, e é de um grande grupo de fótons, o que dificulta ainda mais a constatação de que apenas o elemento tem este comportamento, devido a dificuldade de realização do experimento com apenas um fóton.

Outro fator que veio a induzir erros em nossos resultado é a necessidade de uma distancia grande para que o sensor de luminosidade não sofresse interferências de franjas de luminosidade adjacentes. Em decorrer disto, nossa medida estava com um erro de cerca de $\pm1cm$ da distancia do laser ate a superfície de observação, o que poderia vir a interferir e muito nos valores calculados. Infelizmente, o range que abrangia o erro na amostra para a fenda simples por exemplo, abrangia todo o espectro visível, mesmo estando centrado na coloração correta - verde. Para fins didáticos podemos tomar que os valores estão corretos, já que engloba o espectro visível, porém para fins científicos, seria fundamental que o experimento fosse realizado com ferramentas de precisão, tanto para a verificação das medidas, como para a coleta da intensidade de luz, já que este processo de passar o sensor na frente do feixe de luz foi feito de forma manual e sem grande precisão para manter a velocidade do sensor de forma uniforme.
