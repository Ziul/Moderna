\end{multicols}
\section{Exemplo}\label{eg}
\begin{multicols}{2}    % 2 columns


\IEEEPARstart{C}{omo} dito em \cite{dorf}, a sessão \ref{eg} não tem o menor sentido, assim como a Figura \ref{ce1} ou no código \ref{main}

Podemos também incluir uma citação babaca tipo:
\begin{quote}
	\raggedright
	\textit{"Anyone who has never made a mistake has never tried anything new."}\\
	\raggedleft
	Albert Einstein
\end{quote}

\begin{figure}[H]
\begin{center}
\begin{tikzpicture}[decoration=penciline, decorate]
%  \draw[decorate,style=help lines] (-2,-2) grid[step=1cm] (4,4);
	\draw[decorate,thick,pattern=north east lines] (-0.5cm,0.9)   rectangle (5.5,0.5); % barra de baixo
	\draw[decorate,thick,pattern=north east lines] (-0.5cm,4.4)   rectangle (5.5,4.1); % barra de cima
	\draw[decorate,thick,fill=yellow] (0,0) -- (0,5) -- (2,5) -- (2,0) -- (0,0); % primeira chapa
	\draw[decorate,thick,fill=yellow] (3,0) -- (3,5) -- (5,5) -- (5,0) -- (3,0); % segudna chapa
%	\draw[decorate,thick,fill=red] (2.4,0) -- (2.4,5) -- (2.6,5) -- (2.6,0) -- (2.4,0); % arame
	\draw[decorate,|-|] (2.1,-0.3cm) -- (2.9,-0.3cm); % distancia da primeira fenda
%	\draw[decorate,|-|] (2.7,-0.3cm) -- (3,-0.3cm); % distancia da segunda fenda
%	\draw[decorate,|-|,dashed] (2.7,-0.3cm) -- (3,-0.3cm); % distancia da segunda fenda
	% notas
	\node[black,scale=0.6] at (1,2.5) {Chapa 1};
	\node[black,scale=0.6] at (4,2.5) {Chapa 2};
%	\node[black,scale=0.6] at (2.5,5.2) {Arame};
	\node[black,scale=0.6] at (-1.5cm,0.7) {Barra de fixação ->};
	\node[black,scale=0.6] at (-1.5cm,4.2) {Barra de fixação ->};
	\node[black,scale=0.8] at (2.5,-0.5cm) {$\lambda$};
%	\node[black,scale=0.8] at (2.85,-0.5cm) {$\lambda$};
\end{tikzpicture}
\end{center}
\caption{Fenda simples}
\label{fenda_simples}
\end{figure}

Podemos também incluir algo do código tipo: \lstinline [basicstyle=\ttfamily]{printf("oi!")} ou assim:

\begin{lstlisting}[basicstyle=\ttfamily]
	startup(9600);
	printf("Debug preparado:\n");
\end{lstlisting}

\lipsum

\input graph
