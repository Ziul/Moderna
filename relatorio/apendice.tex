\section{Códigos Fontes}\label{src}

%------------------------------------------------------------------------------%		
% Anexar source main.c com link
%\hypertarget{main}{
%\lstinputlisting[language=C]{../main.c} }
%anexar source main.c sem link
%\lstinputlisting[language=C]{../main.c}
%------------------------------------------------------------------------------%		
%\cite{ti_exemplos}
%\tableofcontents %Índice de conteúdos
%\listoftables %Lista de tabelas
%\listoffigures %Lista de figuras
%------------------------------------------------------------------------------%		
\begin{multicols}{2}    % 2 columns
%caption={[ short ] long }

\hypertarget{main}{
\lstinputlisting[language=C,caption={[main.c]Código principal},label=main]{../trunk/main.c} }

\end{multicols}

\onecolumn
\section{Anexos}\label{anexo}

%Inclusão de Figuras
%--------------------Figura logo da UnB----------------------------------------%
%\begin{figure}[h]
%	\centering
%	\includegraphics [scale=1,angle=0,keepaspectratio=true]{./fts/unb}
%	\caption{Logo da UnB}
%	\label{unb}
%\end{figure}
%--------------------Figura logo da UnB----------------------------------------%
%\begin{SCfigure}[1][h] %%Cuidado..ela não se mantem no lugar
%  \centering
%  \includegraphics[width=4cm]{./fts/unb}
%  \caption{Logo da UnB.}
%  \label{unb1}
%\end{SCfigure}


%\begin{multicols}{4}
\tikzstyle{cloud} = [draw, ellipse,fill=blue!20, node distance=3cm,
    minimum height=2em]
\tikzstyle{estado} = [draw, rectangle,fill=red!20, node distance=3cm,align=left]
\tikzstyle{teste} = [draw, diamond,fill=green!20, node distance=3cm,align=left]
%    minimum height=2em]
\tikzstyle{phanton} = []   
\tikzstyle{line} = [->,right] %[draw, -latex']
\tikzstyle{retorno} = [loop above]
\tikzstyle{arrow} = [bend left,->]


\begin{center}
\begin{tikzpicture}[node distance = 2cm]
	\tiny\ttfamily
	%-- Estados
	\node [cloud] (init) at(0,0) {Inicio};
	\node [estado] (E1) at(2,0) {Inicializa\\ UART e ADC};
	\node [estado] (E2) at(4,0) {Captura ADC};
	\node [estado] (E3) at(4,-2) {Escreve ADC na UART};
	%-- Setas
	\path (init) edge [line] (E1);
	\path (E1) edge [line] (E2);
	\path (E2) edge [bend right,->] (E3);
	\path (E3) edge [bend right,->] (E2);
\end{tikzpicture}
\\\hypertarget{diagrama}{Fluxograma do microcontrolador}
\end{center}


\begin{center}
\begin{tikzpicture}[node distance = 2cm]
	\tiny\ttfamily
	%-- Estados
	\node [cloud] (init) at(0,0) {Inicio};
	\node [estado] (E1) at(1,1) {Inicializa\\ serial};
	\node [estado] (E2) at(3,1) {Abre/Cria\\ arquivos};
	\node [teste] (E3) at(5,1) {interrupção\\ do teclado};
	\node [estado] (E4) at(7,1) {Fecha arquivo\\ e serial};
	\node [estado] (E5) at(5,-1) {lê string\\da serial};
	\node [estado] (E6) at(5,-3) {escreve string\\ no arquivo};
	\node [cloud] (fim) at(7,-1) {fim};
	%-- Setas
	\path (init) edge [line] (E1);
	\path (E1) edge [line] (E2);
	\path (E2) edge [line] (E3);
%	\path (E3) edge [line]   (E4); % S
	\draw [line] node at(5.8,1.1) {S} (E3) --  (E4);
%	\path (E3) edge [line] (E5); % N
	\draw [line] node at(5,0.1) {N} (E3) --  (E5);
	\path (E5) edge [line] (E6);
	\path (E6) edge [arrow] (E3);
	\path (E4) edge [line] (fim);
	
\end{tikzpicture}
\\\hypertarget{diagrama}{Fluxograma do computador}
\end{center}


\begin{figure}[H]
\begin{center}\begin{circuitikz} \draw
	(0,0) node[anchor=east]{$V_{ref}$}
	(6,0) node[anchor=west]{$Ad_0$}
	(0,0) to [R=$R_1$,*-] (3,3) 
	(6,0) to [pR=$LDR$, v=$v_1$,*-] (0,0)
	(3,-3) to [R=$R_2$,*-] (0,0)
	(4.5,-1.5) to [R=$R_3$,*-] (3,-3)
	(6,0) to [pR=$R_4$,*-*] (4.5,-1.5)
	(3,3) to [R=$R_5$,*-*] (6,0)
	(3,-3) node[ground] {};
\draw (3,3) -- node[] {} (3,3.5);
\draw (2.5,3.5) --  node[anchor=south] {VCC} (3.5,3.5)  node[anchor=west] {\SI{5}{V}};

; \end{circuitikz} \end{center}
\caption{Circuito do LDR}
\label{ce1}
\end{figure}


%\end{multicols}

