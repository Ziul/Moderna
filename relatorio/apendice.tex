\section{Códigos Fontes}\label{src}

%------------------------------------------------------------------------------%		
% Anexar source main.c com link
%\hypertarget{main}{
%\lstinputlisting[language=C]{../main.c} }
%anexar source main.c sem link
%\lstinputlisting[language=C]{../main.c}
%------------------------------------------------------------------------------%		
%\cite{ti_exemplos}
%\tableofcontents %Índice de conteúdos
%\listoftables %Lista de tabelas
%\listoffigures %Lista de figuras
%------------------------------------------------------------------------------%		
\begin{multicols}{2}    % 2 columns
%caption={[ short ] long }

\hypertarget{main}{
\lstinputlisting[language=C,caption={[main.c]Código principal},label=main]{../trunk/main.c} }

\end{multicols}

\onecolumn
\section{Anexos}\label{anexo}

%Inclusão de Figuras
%--------------------Figura logo da UnB----------------------------------------%
%\begin{figure}[h]
%	\centering
%	\includegraphics [scale=1,angle=0,keepaspectratio=true]{./fts/unb}
%	\caption{Logo da UnB}
%	\label{unb}
%\end{figure}
%--------------------Figura logo da UnB----------------------------------------%
%\begin{SCfigure}[1][h] %%Cuidado..ela não se mantem no lugar
%  \centering
%  \includegraphics[width=4cm]{./fts/unb}
%  \caption{Logo da UnB.}
%  \label{unb1}
%\end{SCfigure}


%\begin{multicols}{4}
\tikzstyle{cloud} = [draw, ellipse,fill=blue!20, node distance=3cm,
    minimum height=2em]
\tikzstyle{estado} = [draw, rectangle,fill=red!20, node distance=3cm,align=left]
\tikzstyle{teste} = [draw, diamond,fill=green!20, node distance=3cm,align=left]
%    minimum height=2em]
\tikzstyle{phanton} = []   
\tikzstyle{line} = [->,right] %[draw, -latex']
\tikzstyle{retorno} = [loop above]
\tikzstyle{arrow} = [bend left,->]


\begin{center}
\begin{tikzpicture}[node distance = 2cm]
	\tiny\ttfamily
	%-- Estados
	\node [cloud] (init) at(0,0) {Inicio};
	\node [estado] (E1) at(2,0) {Inicializa\\ UART e ADC};
	\node [estado] (E2) at(4,0) {Captura ADC};
	\node [estado] (E3) at(4,-2) {Escreve ADC na UART};
	%-- Setas
	\path (init) edge [line] (E1);
	\path (E1) edge [line] (E2);
	\path (E2) edge [bend right,->] (E3);
	\path (E3) edge [bend right,->] (E2);
\end{tikzpicture}
\\\hypertarget{diagrama}{Fluxograma do microcontrolador}
\end{center}


\begin{center}
\begin{tikzpicture}[node distance = 2cm]
	\tiny\ttfamily
	%-- Estados
	\node [cloud] (init) at(0,0) {Inicio};
	\node [estado] (E1) at(1,1) {Inicializa\\ serial};
	\node [estado] (E2) at(3,1) {Abre/Cria\\ arquivos};
	\node [teste] (E3) at(5,1) {interrupção\\ do teclado};
	\node [estado] (E4) at(7,1) {Fecha arquivo\\ e serial};
	\node [estado] (E5) at(5,-1) {lê string\\da serial};
	\node [estado] (E6) at(5,-3) {escreve string\\ no arquivo};
	\node [cloud] (fim) at(7,-1) {fim};
	%-- Setas
	\path (init) edge [line] (E1);
	\path (E1) edge [line] (E2);
	\path (E2) edge [line] (E3);
%	\path (E3) edge [line]   (E4); % S
	\draw [line] node at(5.8,1.1) {S} (E3) --  (E4);
%	\path (E3) edge [line] (E5); % N
	\draw [line] node at(5,0.1) {N} (E3) --  (E5);
	\path (E5) edge [line] (E6);
	\path (E6) edge [arrow] (E3);
	\path (E4) edge [line] (fim);
	
\end{tikzpicture}
\\\hypertarget{diagrama}{Fluxograma do computador}
\end{center}


\begin{figure}[H]
\begin{center}\begin{circuitikz} \draw
	(0,0) node[anchor=east]{$V_{ref}$}
	(6,0) node[anchor=west]{$Ad_0$}
	(0,0) to [R=$R_1$,*-] (3,3) 
	(6,0) to [pR=$LDR$, v=$v_1$,*-] (0,0)
	(3,-3) to [R=$R_2$,*-] (0,0)
	(4.5,-1.5) to [R=$R_3$,*-] (3,-3)
	(6,0) to [pR=$R_4$,*-*] (4.5,-1.5)
	(3,3) to [R=$R_5$,*-*] (6,0)
	(3,-3) node[ground] {};
\draw (3,3) -- node[] {} (3,3.5);
\draw (2.5,3.5) --  node[anchor=south] {VCC} (3.5,3.5)  node[anchor=west] {\SI{5}{V}};

; \end{circuitikz} \end{center}
\caption{Circuito do LDR}
\label{ce1}
\end{figure}

\begin{figure}[H]
\begin{center}
\begin{tikzpicture}[decoration=penciline, decorate,scale=2]
%  \draw[decorate,style=help lines] (-2,-2) grid[step=1cm] (4,4);
	\draw[decorate,thick,pattern=north east lines] (-0.5cm,0.9)   rectangle (5.5,0.5); % barra de baixo
	\draw[decorate,thick,pattern=north east lines] (-0.5cm,4.4)   rectangle (5.5,4.1); % barra de cima
	\draw[decorate,thick,fill=yellow] (0,0) -- (0,5) -- (2,5) -- (2,0) -- (0,0); % primeira chapa
	\draw[decorate,thick,fill=yellow] (3,0) -- (3,5) -- (5,5) -- (5,0) -- (3,0); % segudna chapa
	\draw[decorate,thick,fill=red] (2.4,0) -- (2.4,5) -- (2.6,5) -- (2.6,0) -- (2.4,0); % arame
	\draw[decorate,|-|] (2,-0.3cm) -- (2.3,-0.3cm); % distancia da primeira fenda
	\draw[decorate,|-|] (2.7,-0.3cm) -- (3,-0.3cm); % distancia da segunda fenda
%	\draw[decorate,|-|,dashed] (2.7,-0.3cm) -- (3,-0.3cm); % distancia da segunda fenda
	% notas
	\node[black,scale=0.6] at (1,2.5) {Chapa 1};
	\node[black,scale=0.6] at (4,2.5) {Chapa 2};
	\node[black,scale=0.6] at (2.5,5.2) {Arame};
	\node[black,scale=0.6] at (-1cm,0.7) {Barra de fixação ->};
	\node[black,scale=0.6] at (-1cm,4.2) {Barra de fixação ->};
	\node[black,scale=0.8] at (2.15,-0.5cm) {$\lambda$};
	\node[black,scale=0.8] at (2.85,-0.5cm) {$\lambda$};
\end{tikzpicture}
\end{center}
\caption{Fenda dupla}
\label{fenda_dupla}
\end{figure}

\begin{figure}[H]
\begin{center}
\begin{tikzpicture}[decoration=penciline, decorate,scale=2]
	\draw[thick] (0,0) -- (0,2.4) ; % barra de baixo
	\draw[thick] (0,2.6) -- (0,5) ; % barra de cima
	\draw[thick] (3,0) -- (3,5) ; % barra de choque
	\draw[line] (-1,2.5) -- (-0.2,2.5) ; % barra de choque
	% ondas primarias
	\foreach \A in {0,0.1,0.2,0.3,0.4,0.5} {
		\draw[thick] (-0.8 + \A,2) -- (-0.8 + \A,3) ;
	}
	\node[black,scale=0.8] at (-0.6,1.8) {$\overbrace{\text{Ondas incidentes}}$};
	% ondas após a fenda
	\foreach \A in {0,0.2,0.4,0.6,0.8,1,1.2,1.4} {
		\draw[thick] (0.2+\A,2.0-\A) -- (0.2+\A,2.01-\A) arc (-70:70:0.5+\A) ;
	}
\end{tikzpicture}
\end{center}
\caption{Difração em fenda simples}
\label{diff_simples}
\end{figure}


%--------------------------------
\begin{figure}[H]
\begin{center}
\begin{tikzpicture}[decoration=penciline, decorate,scale=2]
	\draw[thick] (0,0) -- (0,2.2) ; % barra de baixo
	\draw[thick] (0,2.4) -- (0,2.6) ; % barra de baixo
	\draw[thick] (0,2.8) -- (0,5) ; % barra de cima
	\draw[thick] (3,0) -- (3,5) ; % barra de choque
	\draw[line] (-1,2.5) -- (-0.2,2.5) ; % barra de choque
	% ondas primarias
	\foreach \A in {0,0.1,0.2,0.3,0.4,0.5} {
		\draw[thick] (-0.8 + \A,2) -- (-0.8 + \A,3) ;
	}
	\node[black,scale=0.8] at (-0.6,1.8) {$\overbrace{\text{Ondas incidentes}}$};
	% ondas após a fenda
	\foreach \A in {0,0.2,0.4,0.6,0.8,1,1.2,1.4,1.6} {
		\draw[thick,color=red] (0.1+\A,2-\A) -- (0.1+\A,2.01-\A) arc (-70:70:0.3+\A) ;
	}
	% ondas após a fenda
	\foreach \A in {0,0.2,0.4,0.6,0.8,1,1.2,1.4,1.6} {
		\draw[thick,color=green] (0.1+\A,2.4-\A) -- (0.1+\A,2.41-\A) arc (-70:70:0.3+\A) ;
	}
	% colisão das ondas
	\draw[dashed,->] (0,2.4) -- (5:3.5cm) ; % colisão 1
	\draw[dashed,->] (0,2.6) -- (50:5.5cm) ; % colisão 2
	\draw[dashed,->] (0,2.5) -- (34:4.3) ; % colisão 3
	
	\node[black,scale=0.8] at (4.2,2.4) {Colisão das ondas};
	\node[black,scale=0.8] at (4.2,4.2) {Colisão das ondas};
	\node[black,scale=0.8] at (4.16,0.22) {Colisão das ondas};
\end{tikzpicture}
\end{center}
\caption{Difração em fenda dupla}
\label{diff_dupla}
\end{figure}


%\end{multicols}

