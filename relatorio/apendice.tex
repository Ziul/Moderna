\section{Códigos Fontes}\label{src}

%------------------------------------------------------------------------------%		
% Anexar source main.c com link
%\hypertarget{main}{
%\lstinputlisting[language=C]{../main.c} }
%anexar source main.c sem link
%\lstinputlisting[language=C]{../main.c}
%------------------------------------------------------------------------------%		
%\cite{ti_exemplos}
%\tableofcontents %Índice de conteúdos
%\listoftables %Lista de tabelas
%\listoffigures %Lista de figuras
%------------------------------------------------------------------------------%		
\begin{multicols}{2}    % 2 columns
%caption={[ short ] long }

\hypertarget{main}{
\lstinputlisting[language=C,caption={[main.c]Código principal},label=main]{../trunk/main.c} }

\end{multicols}

\onecolumn
\section{Anexos}\label{anexo}

%Inclusão de Figuras
%--------------------Figura logo da UnB----------------------------------------%
%\begin{figure}[h]
%	\centering
%	\includegraphics [scale=1,angle=0,keepaspectratio=true]{./fts/unb}
%	\caption{Logo da UnB}
%	\label{unb}
%\end{figure}
%--------------------Figura logo da UnB----------------------------------------%
%\begin{SCfigure}[1][h] %%Cuidado..ela não se mantem no lugar
%  \centering
%  \includegraphics[width=4cm]{./fts/unb}
%  \caption{Logo da UnB.}
%  \label{unb1}
%\end{SCfigure}


%\begin{multicols}{4}
\tikzstyle{cloud} = [draw, ellipse,fill=red!20, node distance=3cm,
    minimum height=2em]
\tikzstyle{phanton} = []   
\tikzstyle{line} = [->,bend left] %[draw, -latex']
\tikzstyle{arrow} = [loop above]

\begin{center}
\begin{tikzpicture}[node distance = 2cm]
	\tiny\ttfamily
	%-- Estados
	\node [cloud] (E0) at(0,2) {E0};
	\node [cloud] (E1) at(2,1) {E1};
	\node [cloud] (E2) at(2,-1) {E2};
	\node [cloud] (E3) at(0,-2) {E3};
	\node [cloud] (E4) at(-2,-1) {E4};
	\node [cloud] (E5) at(-2,1) {E5};
	%-- setas
	\path (E0) edge [line] (E1);
	\path (E1) edge [line] (E2);
	\path (E2) edge [line] (E3);
	\path (E3) edge [line] (E4);
	\path (E4) edge [line] (E5);
	\path (E5) edge [line] (E0);
	%-- Desvios
	\path (E0) edge [draw,loop above] node{controle1=1} (E0);
	\path (E3) edge [draw,loop below] node{controle2=1} (E3);
	\path (E2) edge [bend right,->]  node[anchor=east]{controle1=1\&\&controle2=0} (E0);
	\path (E5) edge [bend right,->]  node[anchor=west]{controle1=0\&\&controle2=1} (E3);
\end{tikzpicture}
\\\hypertarget{diagrama}{Diagrama de Estados}
\end{center}

\begin{circuitikz}[american, scale = 2] \draw
(0,0) node[op amp] (opamp) {}
(opamp.+) node[left ] {$v_+$}
(opamp.-) node[left] {$v_-$}
(opamp.out) node[right] {$v_o$}
(opamp.down) node[ground] {}
(opamp.up) ++ (0,.5) node[above] { {12}{V}} -- (opamp.up)
;\end{circuitikz}



\begin{figure}[H]
\begin{center}\begin{circuitikz} \draw
	(0,0) to [R=$R_1$,*-] (2,0) --
	(2,0) to [R=$R_1$, i=$i _1$, v=$v_1$] (4,0) --
	(4,0) to [C=$R_1$, i=$i _1$, v=$v_1$] (4,2) --
	(4,2) to [R, -*] (1,2)
	(0,4) to [ sI=$a_1$] (2,4)
	(3,4) to [ I=$a_1$] (5,4)
; \end{circuitikz} \end{center}
\caption{circuito perdido}
\label{ce1}
\end{figure}

%\end{multicols}

