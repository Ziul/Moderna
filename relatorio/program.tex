\section{Levantando o programa}\label{program}

Inicialmente, é importante frisar que o kit do Arduíno nada mais é que uma camada de abstração criada acima do microcontrolador AVR. Desta forma, podemos programar nos kits do Arduíno de duas formas sob o sistema Linux:

\subsection{Shield Comum}\label{noob}

A forma mais simples de utilizar um kit do Arduíno é utilizando a própria IDE oferecida para o kit. A sua instalação é bem simples e fácil de ser feita. Basta entrar com o seguinte comando no terminal:


%\begin{lstlisting}[basicstyle=\ttfamily,numbers=none,caption={[exemplo]Código de exemplo},style=Bash]  % Assim fica de um tamanho aceitável e sem os números das linhas do lado
%	sudo apt-get install arduino
%\end{lstlisting}

\begin{lstlisting}[style=Bash]
user@DESKTOP: sudo apt-get install arduino
\end{lstlisting}
