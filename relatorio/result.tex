\section{Resultados}\label{result}

Para a verificação dos resultados, são necessárias, em suma, duas preocupações principais. A primeira é de interpretação da teoria quantica da luz, e perceber seu comportamento de onda. A segunda é a verificação prática do comprimento de onda do laser utilizado, por meio das medidas e dados coletados. Se o comprimento de onda corresponder com a cor de laser usado no experimento, o experimento passa a ter um resultado positivo.

\subsection{Fenda Simples}\label{fenda_simples}
\begin{equation}
	\lambda \approx \frac{ya}{mD}
\end{equation}
Sendo $a$ o diâmetro da fenda, $m$ a quantidade de pontos escuros a partir do centro, $D$ a distância da fenda até o ponto de leitura, e $y$ a distância do centro luminoso até ultimo ponto de leitura.

%soh esqueleto pra continuar dps
\subsection{Fenda Dupla}\label{fenda_dupla}
\begin{equation}
	\lambda \approx \frac{yd}{mD}
\end{equation}
Sendo $d$ o diâmetro do conjunto fenda-fenda, $m$ a quantidade de pontos escuros a partir do centro, $D$ a distância da fenda até o ponto de leitura, e $y$ a distância do centro luminoso até ultimo ponto de leitura.
%soh esqueleto pra continuar dps